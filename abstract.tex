\chapter*{Abstract}

\noindent

21 cm observations promise to open a new window in our understanding of the universe. However, that come at high price. 21 cm observation require instruments with high sensitivity and a large field of view. Furthermore, robust techniques are require to do a high precision calibration and  foreground removal. In this work, we investigate how effeciently can linear redundant calibration method recover antenma gain factors and true sky signal in the case of  i) Perfect Redundant Array and ii) Quasi-Redundant Array .In the case of perfect redundancy, Lincal is able to recover approximately all simulated corrections within only a few iterations. However, in the case of quasi-redundancy, due to either antenna primary beams imperfections or location variations, Lincal is unable to perfectly recover the simulated data. An alternative calibration method is introduced, which aims to drastically improve upon the results of Lincal, with focus on the quasi-redundancy in antenna layout. The correlation calibration scheme, includes our knowledge of the statistical properties of the sky and the known bright sources in the calibration analysis. This correlation calibration scheme will be used to calibrate data for the future 21 cm instruments such as Hydrogen Intensity and Real-time Analysis eXperiment (HIRAX), Canadian Hydrogen Intensity Mapping Experiment (CHIME) and the  SKA Telescope .\\

