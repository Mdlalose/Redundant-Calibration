\chapter{Introduction}

Despite the great progress that has been made in understanding cosmology, the history of the universe between recombination and the $z \sim 6$ is almost unconstrained observationally. Observations of the 21 cm line from HI opens a new window in cosmology to study structure formation at high redshift and it will help to constrain cosmological parameters\citet{a}. However, 21 cm signal is very weak compare  to the foreground signal, which mainly come from milky galaxy ( 4 oders of magnitude brighter) \citet{b} and other bright extragalactic sources. In addition to foreground noise, the earth's atmosphere and instrument systematic errors contaminate 21 cm signal. These problems on 21 cm observations require robust techniques to calibrate precisely the instrument and to remove foregrounds signal. \\
 A new generation of instruments with high sensitivity and larger field of view are required to meet such demands for 21 cm observations.  21 cm instruments such Precision Array to Probe the Epoch of Reionization (PAPER) and the upcoming instruments such as Hydrogen Epoch of Reionization Array (HERA)\citet{a},Hydrogen Intensity and Real-time Analysis eXperiment  (HIRAX) and Canada Hydrogen Intensity Mapping Experiment (CHIME) are generic redundant in their design. This redundancy feature is the keypoint in making a calibration analysis more computation efficeintly.
 \\In this work, we invistigate how efficient can a linear redundant calibration method in a case of i) perfect redundant array and ii) quasi redundant array. We introduce altinative method, called correlation calibration, which take into account the impeferction in any array. In section 1.1, we briefly discuss the basic 21 cm physics.  Linear redundant calibration formation is discussed in Chapter 2. In section 2.2 we briefly discuss correlation calibration. In chapter 3, we discuss the results. Lastly, in chapter we provide a conclusion and future work map.

Example citation... \citet{Abell_1958}.

